\section{Related Work}
\label{sec:sec002}

This section addresses related work in the HCI field, describing several MI applications.
Our approach covers the limitations of the works following described.

\subsection{Data Visualization}

To our knowledge, few papers~\cite{10.1145/1133265.1133354, 10.1145/2909132.2909248, 10.1145/3206505.3206602} have focused purely on supporting the image search user experience through novel UIs.
These authors described several techniques for presenting all images within a collection in a short time.
Moreover, authors asked users to think and perform browsing an image gallery and selecting an image from the gallery.
These studies, showed us refinement techniques as complements in image systems with relevant user feedback.
However, the presented works are limited to non-clinical users, making it impossible to do a generalization to our research.

\footnotetext[1]{\scriptsize This is the common/current practice in the radiologist services applied in the \hyperlink{https://hff.min-saude.pt/}{Hospital Fernando Fonseca} (HFF), Portugal.}

\subsection{Clinical Workflow}

In medical imaging, diagnostic tools enable clinicians to manage patient data, better attend to ongoing tasks and view critical information.
For the diagnostic, understanding the clinical workflow is of chief importance while introducing novel tools and interaction techniques.
Other authors~\cite{10.1145/2685553.2699332, 10.5555/2826165.2826187} present many considerations for collaborative healthcare technology design and discuss the implications of their findings on the current clinical workflow for the development of more effective care interventions.
Supported by the above literature, our goal is to introduce a new tool with several novel interaction techniques, which will improve the final medical imaging diagnosis.

\subsection{Medical Imaging}

From current medical imaging technologies, several issues were identified in the HCI design~\cite{Calisto:2017:TTM:3132272.3134111, Igarashi:2016:IVS:2984511.2984537}.
Some works~\cite{Balducci:2018:BQA:3206505.3206555, Rosado:2015:NFS:2826165.2826213} show the current medical imaging identification techniques for other clinical domains, where most of available systems fail to address the visual nature of the task.
In these two works~\cite{Balducci:2018:BQA:3206505.3206555, Rosado:2015:NFS:2826165.2826213}, the authors create a visual approach to support the \textit{Mental Model} development of the user.
Medical imaging technologies are used to support physicians on the \textit{examination}, \textit{diagnosis}, and (in some cases) \textit{report}~\cite{10.1145/2639189.2639256}.
Others~\cite{10.1145/1385569.1385651, 10.1145/1842993.1843023}, study the effectiveness and performance of medical imaging systems, demonstrating how to design a user study for medical imaging experts.
Further, van Schooten et al.~\cite{10.1145/1842993.1843023} measured user performance in terms of time taken and error rate, while interacting with the provided system.
Executing it with several medical users, in this work, the authors show an experiment where their users have similar characteristics as ours.

\subsection{Diagnostic Systems}

Medical imaging has also been extensively studied under the topic of \textit{Computer-Aided Diagnosis (CADx)}, which refers to systems that assist radiologists in image interpretation~\cite{Oram:2014:CDR:2598510.2598585, Molin:2016:UDA:2971485.2971561}.
Wilcox et al.~\cite{Wilcox:2010:DPI:1753326.1753650} propose a design for in-room, patient-centric information displays, based on iterative design with clinicians.
However, these systems are not contemplating the design of an advanced visual interface for multimodal diagnosis on breast cancer disease.
In the above works, we still lack on empirical studies regarding how clinicians can contribute with information contextualization about their clinical workflow, and general medical imaging diagnosis.
Having said that, we also want to add contribution with a study of how medical imaging technologies can play a role in this contextualization.