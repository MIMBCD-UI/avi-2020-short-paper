\section{Design of BreastScreening}
\label{sec:sec003}

The design of \href{https://breastscreening.github.io/}{{\it BreastScreening}} started with a qualitative study to understand radiology practices and workflow in the context of breast screening.
Our study involved 31 clinicians, recruited on a volunteer basis from a large range of clinical scenarios (distinct health institutions in Portugal):
8 clinicians from Hospital Fernando Fonseca; % HFF
12 clinicians from IPO-Lisboa; %IPOL
1 clinician from Hospital de Santa Maria; %HSM
8 clinicians from IPO-Coimbra; %IPOC
1 clinician from Madeira Medical Center; and %MMC
1 clinician from SAMS. %SAMS
Clinicians' experience ranged from 5 - 30 years of medical practice.
The recruited specialists are in advanced career positions and were observed and interviewed in a semi-structured fashion.
Each session took approximately 30 minutes.

\subsection{Standard Clinical Environments}

\href{https://breastscreening.github.io/}{{\it BreastScreening}} works with the standard formats supported by medical imaging~\cite{ng2017technical}, including the MG, US and MRI modalities.
These modalities are available in a standard  Digital Imaging and Communications in Medicine (DICOM) format and supported in Single-Modality by existing systems~\cite{henriksen2018efficacy}. Moreover, most systems are general purpose and do not adapt to specific clinical domains ({\it e.g.}, breast screening). Therefore they do not provide adequate support to the different clinical workflows~\cite{Calisto:2017:TTM:3132272.3134111}.

\subsection{Design Goals}

Combining the clinical context and the technical design challenges lead to a set of design issues, including: \textit{medical imaging structure trade-offs}, \textit{RR temporal awareness}, \textit{image segmentation}~\cite{8736792}, and radiologists system trust.
Based on these, we define five design goals:
\begin{description}
\item[{\it \underline{D}esign around and for \underline{M}edical \underline{I}maging} (DMI):] by taking into account the heterogeneous nature of medical imaging to leverage its contextual richness;
\item[{\it \underline{T}emporal \underline{A}wareness \underline{S}upport} (TAS):] by observing how the radiology workflow events, treatments, and problems progressed over time;
\item[{\it \underline{I}mage \underline{S}egmentation \underline{S}upport} (ISS):] the overview of image details allowing a more accurate diagnostic. Namely, reducing the number of false-positives classification (BIRADS) of the lesion, as well as improving the number of clicks (Section \ref{sec:sec005}) when performing the lesion delineation, {\em i.e.}, segmentation;
\item[{\it\underline{S}everal \underline{M}odalities \underline{S}upport} (SMS):] to enable the view and the process of diagnostic imaging studies, including MG, US and MRI medical imaging modalities;
\item[{\it\underline{G}rowing \underline{T}rust \underline{O}verview} (GTO):] by allowing an efficient triangulation via visualizations, image processing between medical images and available features, {\em i.e.}, annotations of masses and calcifications;
\end{description}
