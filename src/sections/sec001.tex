\section{Introduction}
\label{sec:sec001}

Breast cancer is the most common cancer in women worldwide~\cite{henriksen2018efficacy}.
Screening plays a fundamental role in the reduction of patient mortality rate.
The most widely employed image modality for breast screening is MammoGraphy (MG).
However, high-risk or dense breast patients require UltraSound (US) or Magnetic Resonance Imaging\footnotemark[1] (MRI) for proper examination~\cite{Maicas2019}.
Therefore, it is quite rare to conduct screening using a \textit{Single-Modality}.

In this paper, we describe the design and comparative testing of \href{https://breastscreening.github.io/}{{\it BreastScreening}} integrating information from several and different image modalities.
We tested the design of \href{https://breastscreening.github.io/}{{\it BreastScreening}} with 31 clinicians noting that the time spent per each image on a \textit{Multi-Modality} strategy is reduced when compared with the \textit{Single-Modality} scenario.
In addition, the lesion classification ({\em e.g.}, \href{https://breast-cancer.ca/bi-rads/}{Breast Imaging Reporting and Data System - BIRADS} ~\cite{SPAK2017179}) is also reduced from our \textit{Multi-Modality} proposed approach.

\subsection{BreastScreening Challenges}

Overall the system involves the  following functionalities:
(1) an interface for identifying (and annotating ground truth) of two types of lesions (i.e., masses and calcifications) across image modalities;
(2) support for categorization of the breast tissues (dense vs non-dense);
(3) a classification (and recommendation) schema for lesion severity using \href{https://breast-cancer.ca/bi-rads/}{BIRADS}~\cite{aghaei2018association, SPAK2017179};
(4) prompt access to clinical co-variables, such as personal and familiar records; and
(5) proper visualizations for a follow-up diagnosis of the patients.

\subsection{Design Process}

The following topics summarize the process we conducted:
(1) findings from a formative study with 31 clinicians, comprising Radiology Room (RR) observations and interviews, which are relevant for both Health Informatics (HI) and Human-Computer Interaction (HCI) fields of research.
This leads us to explore the {\it design goals} (see Section~\ref{sec:sec003});
(2) findings from an evaluation study~\cite{https://doi.org/10.13140/rg.2.2.16566.14403/1} of \href{https://breastscreening.github.io/}{{\it BreastScreening}}, a prototype we developed for the generation of a breast dataset with expert annotations (see Section~\ref{sec:sec004}); and
(3) design recommendations for the use of visualizations to support medical imaging diagnosis (see Sections~\ref{sec:sec004} and \ref{sec:sec005}).

\subsection{Contributions}

In \href{https://breastscreening.github.io/}{{\it BreastScreening}} we provide several new insights, following novel interaction and visualization paradigms~\cite{PAULO2019103316} in the context of breast cancer screening:
$(i)$ multimodal interaction;
$(ii)$ indistinct visualization of cluttered lesions;
$(iii)$ big data management platform; and
$(iv)$ clinicians' multi-screen, multi-environment interaction.